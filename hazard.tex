
\section{Seismic Hazard Approach}

There have been multiple assessments of seismic hazard for Iran. A brief report on the most recent review done to define the hazard zoning map of Iran included in the Iranian seismic code \citep{BHRC2014} can be found in \citet{Moinfar_2012_WCEE}. Other previous efforts include, for instance, the work of \citet{Tavakoli1999}, who presented iso-acceleration contour lines and seismic hazard zonation for Iran based on known major fault-lines and area-source models; the results obtained by \citet{Ghodrati2003}, who presented a refined hazard assessment for the metropolitan area of Tehran using a logic tree approach to combine results from three attenuation relationships; and the recent study by \citet{Khodaverdian_2016_BSSA}, who used a smoothed seismicity approach to estimate spatially variable seismicity parameters for the whole country.

Once a seismic catalog has been defined, the second most important step is precisely the definition of the seismicity parameters $a$ and $b$ used in the Gutenberg-Richter recurrence relation 
% 
\begin{equation}
	\log_{10} N = a - b M \, ,
\end{equation}
% 
\noindent
where $N$ is the number of earthquakes with magnitude greater than or equal to $M$, $a$ accounts for the number of earthquakes that occur within a given region, and $b$ is a measure of the occurrence of earthquakes relative to magnitude \citep{Gutenberg1944}. These parameters can be assumed constant within a region or spatially variable depending on the approach used. 

In this study we use the smoothed seismicity method introduced by \citet{Frankel1995}. This method has been used successfully in different regions around the world including Alaska, California, Hawaii, Italy, Turkey \citep[e.g.,][]{Cao1996, Klein2001, Akinci2004, Kalkan2009, Moschetti2014}. In it, the $b$-value is commonly assumed constant within a given seismic zone of ``homogeneous'' seismogenic characteristics, whereas $a$ is varied spatially according to the cataloged occurrence of earthquakes within and around the region. The spatial variation of $a$ is defined based on dividing the region of interest into regular-size grid cells. The parameter $a$ is determined based on the number of events $n_i$ within each cell $i$ of magnitude greater than a reference value $M_{\mathrm{ref}}$, and discretized in different magnitude ranges (or bins). The number of earthquakes within each cell is then smoothed spatially taking into account the values of neighboring cells using a Gaussian function. In essence, the smoothed seismicity approach addresses the aleatoric uncertainty of future earthquake locations by using a weighted contribution of the background regional seismicity, and thus eliminates the need for knowledge about specific fault locations.

In this study we discretize the region of interest into a regular grid with cells of size 0.1\textdegree{} $\times$ 0.1\textdegree{} (in longitude and latitude). The parameter $a$ for each cell is obtained based on the information in the seismic catalog described in the previous section. Following \citet{Frankel1995}, $n_i$ is smoothed using the normalized Gaussian function
% 
\begin{equation}
	\tilde{n}_i = \frac
		{ \sum_{j} n_{j} \exp ( \frac{ -\Delta_{ij}^{2} }{ c^2 } ) }
		{ \sum_{j} \exp ( \frac{ -\Delta_{ij}^{2} }{ c^2 } ) }
	\, ,
	\label{eq:nnorm}
\end{equation}
% 
where $\tilde{n}_i$ is the smoothed, normalized number of events in cell $i$, $\Delta_{ij}$ is the distance between the $i$-th and $j$-th cells, and $c$ is the correlation distance. The summations in Eq.~\ref{eq:nnorm} are calculated over the number of $j$ cells within a distance $3c$ from the $i$-th cell. It should be noted that in this process we convert the cumulative number of events to incremental values following \citet{Herrmann1977}. We also note that according to \citet{Frankel1995}, moderate earthquakes generally occur in areas where there have been significant numbers of events with magnitudes $M \geq 3$, thus our interest in having complemented the earthquake catalog for events above this threshold.

It is obvious from Eq.~\ref{eq:nnorm} that an important parameter in smoothing the seismicity is the correlation distance. Different values of $c$ have been used in past studies, ranging between 10 and 50 km. \citet{Frankel1995} and \citet{Boyd2008} assumed a correlation distance of 50~km, \citet{Foteva2006} used 10 and 15 km, \citet{Barani2007} used 25~km, and \citet{Khodaverdian_2016_BSSA} used 40 km. According to \citet{Mirzaei1997}, determining an appropriate value of $c$ depends on the crustal structure of the region, the distribution of stations, and the number and quality of records. We use an intermediate value of $c=30$~km. This selection is consistent with epicentral uncertainty for previous hazard zoning maps for Iran \citep[see][]{Zare2012}.

In turn, we consider the $b$-values to be constant within each seismic zone as delimited in Fig.~\ref{fig:iran}. Details about the calculation of the $b$-value for each zone are provided below. By contrast, in the recent work by \citet{Khodaverdian_2016_BSSA}, the authors computed spatially-variable $b$-values within large grid-cells of size 1\textdegree{} $\times$ 1\textdegree{} based on the average seismicity within a 200-km radius from each cell. While gridding $b$-values and considering the influence of neighboring large-size cells allows to draw information from the background seismicity, it is unavoidable to have the contributing areas cross over seismic zones of different characteristics, which may pollute the computation of $b$-values. Admittedly, these are subjective decisions and any reasonable choice can be considered valid or within the margins of the associated epistemic uncertainty.

Last, the annual rate $\lambda$ of exceeding a reference ground motion $u_o$ is defined by
% 
\begin{align}
	\lambda & \left( u > u_o \right) = \nonumber \\
		& \sum_{k} \sum_{l} 10^{ {}^{ \left[ \log \left( \frac{ N_{k} }{ T } \right) - b \left( M_l - M_{\mathrm{ref}} \right) \right] } }
		P \left( u > u_o | D_k , M_l \right)
		\, .
	\label{eq:exceed}
\end{align}
% 
Here, the values of $\tilde{n}_i$ have been previously binned by their distance from the site of interest, $N_k$ is the total of $\tilde{n}_i$ values for cells within a certain bin distance, $T$ is the time in years used to determine $N_k$, and $k$ and $l$ are the indexes of the distance and magnitude bins, respectively. $P ( u > u_o | D_k , M_l )$ is the probability of $u$ exceeding $u_o$ given an earthquake of magnitude $M_l$ at a distance $D_k$ \citep{Frankel1995}.

It is clear from Eq.~\ref{eq:exceed} that the hazard calculations depend also on the choice of an attenuation relationship to predict the level of ground motion at a site ($u$) for any given earthquake. In the following two sections we describe how we obtained the $b$-values for the different seismic zones influencing the hazard in northern Iran and our choice of a suitable attenuation relationship (ground motion prediction equation).

