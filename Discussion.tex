
\section{Discussion}

Seismic hazard in northern Iran is computed based on background seismicity and projected on a set of hazard maps. The results obtained are represented as maps for the spatial distribution of horizontal peak ground acceleration with 10\% and 2\% probability of exceedance in 50 years, which correspond to the return period of 475 and 2475 years. The areas of large probabilistic ground motions clearly coincide with zones with a large number of events of magnitude 3.0 and larger. Figure 5 presents the map of PGA for 10\% exceedance in 50 years. The highest and the lowest contour values are 0.44g and 0.12g, respectively. The highest and lowest values in Fig.~\ref{fig:10percent} are 0.74g and 0.19g, respectively. (We disregarded the values in the Caspian Sea and also close to the study region's borders). Highest hazards are established in the Alborz tectonic seismic region in the city of Damavand, about 55 km east of Tehran city in Tehran province. Lowest hazards are established in the Azerbaijan tectonic seismic province, about 31 km east of Shahin Dezh in West Azerbaijan province. Also many other places are in higher hazard zone both in Azerbaijan and and Alborz seismic tectonic regions (See Fig.~\ref{fig:10percent} and Fig.~\ref{fig:2percent}). Many seismic hazard analysis studies have been done in the region based on different methods. Table 3 presents the PGA value for some of the major cities in northern Iran from different studies.

\begin{table}[h]
\centering
\caption{Comparison of PGA, from different studies for selected cities in Northern Iran. (V2009:  \citet{vafaie2011}), G2008: \citet{Ghodrati2008}, G2010: \citet{Ghodrati2010},  Ah2013: \citet{Ahmadi2013}, G2003:  \citet{Ghodrati2003}, G2011: \citet{Ghodrati2011}, Ab2014: \citet{Abdollahzadeh2014} , Ra2012: \citet{Rahgozar2012} , Ak2011: \citet{Akbari2011} ) }
\begin{tabular}{ | c | c | c | c | c | c | c | c | c | c |}


\hline

	
	\multirow{2}{*}{Cities} & \multirow{2}{*}{Lon} & \multirow{2}{*}{Lat} & \multicolumn{2}{|c|}{This study} & \multirow{2}{*}{2800} & Zare & \multicolumn{3}{|c|}{Other Refrences}    \\ 
	\cline{4-5}  \cline{8-10}  &  &  & 10\% & 2\% &  &  2012 & 10\% & 2\% & ref \\ \hline
	Orumyeh & 45.07 & 37.55 & 0.22 & 0.39 & 0.3 & 0.35-0.5 &  &  &  \\ \hline
	Tabriz      & 46.3     & 38.06 & 0.27& 0.50 & 0.35 & 0.35-0.5 & 0.2- 0.65 & 0.3 to 0.9 & V2009 \\ \hline
	Ardabil    & 48.28 & 38.25 & 0.36 & 0.63 & 0.3 & 0.35-0.5 &  &  &  \\ \hline
	Manjil     & 49.40 & 36.74 & 0.39 & 0.71 & 0.35 & 0.65 $<$ & 0.25 & 0.4 & G2008 \\ \hline
	Rasht     & 49.58 & 37.27 & 0.33 & 0.55 & 0.3 & 0.5-0.65 & 0.1 & 0.2 & G2008 \\ \hline 
	Arak       & 49.68 & 34.09 & 0.24 & 0.48 & 0.25 & 0.35-0.5 & 0.16-0.36 & 0.25-0.55 & G2010 \\ \hline
	Qazvin   & 50.00 & 36.26 & 0.27 & 0.50 & 0.35 & 0.35-0.5 & 0.31 & 0.42 & Ah2013 \\ \hline
	 \multirow{2}{*}{Tehran}  & \multirow{2}{*}{51.42} & \multirow{2}{*}{35.69} & \multirow{2}{*}{0.35} & \multirow{2}{*}{0.59} & \multirow{2}{*}{0.35} & \multirow{2}{*}{0.35-0.5} & 0.37-0.415 &  &  G2011 \\ 
	 \cline{8-10}	             &  &  &  &  &  &  & 0.27-0.46 & 0.35 & G2003\\ \hline
	Gorgan & 54.43 & 36.83 & 0.39 & 0.69 & 0.3 & 0.35-0.5 & 0.34 & 0.3$<$ & Ab2014\\ \hline
	Bojnurd & 57.33 & 37.47 & 0.38 & 0.68 & 0.3 & 0.35-0.5 &0.16-0.2  & 0.32-0.45  & Ra2012  \\ \hline
	Mashhad & 59.6 & 36.29 & 0.16 & 0.28 & 0.3 & 0.35-0.5 &  &  & Ak2011 \\ \hline

\end{tabular}

\end{table}


According to Table 3, the results of smoothed seismicity  are comparable with many other studies that considered faults and seismogenic zones. According to \citet{BHRC2014}, our results show that the smoothed seismicity approach gives reasonable regionalized results for 10\% probability of exceedance even when there is no need to introduce seismogenic zones. However, in regions for which the catalog is not complete or the study area is near the fault, probabilistic seismic hazard studies, which are implemented by seismic source, should be conducted. 


