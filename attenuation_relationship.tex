\section{Attenuation relationship}
The choice of a ground motion attenuation model is of great importance since attenuation has proven to be a highly influential factor of seismic hazard. \citet{Zafarani2014} used 163 free-field acceleration time histories recorded at epicentral distance of up to 200 km from 32 earthquakes to investigate the predictive capabilities of the local, regional, and next generation attenuation (NGA) ground-motion prediction equations and determined their applicability for northern Iran. After evaluating different Ground Motion Prediction Equations, \citet{Kalkan2004}, \citet{Chiou2008}, and \citet{Boore2008} represented suitable performance for  PGA  with LLH (Log-Likelihood method) score of 1.54, 1.55, and 1.59. Mean of the mentioned attenuation relationships (not shown here), are very close together, especially at higher magnitude. Using \citet{Scherbaum2009} approach and \citet{Zafarani2014} coefficients we calculated the logic tree weights as 0.3376, 0.3354, and 0.3270, respectively. Getting close mean values and logic tree weights, in order to consider the epistemic uncertainty, instead of using several GMPEs we use the top rank GMPE (i.e.  \citet{Kalkan2004} ) with  $\pm$ standard deviation with weight of 0.2 (for each of standard deviation branches) and 0.6 for the mean value.    \\

\noindent

The attenuation relationship is:

\begin{equation}
ln\ (Y) = b_1 + b_2(M_w-6) + b_3( M_w-6)^{2}+ b_5ln\ r + b_V \ ln(V_S/V_A) \  with \  r= \sqrt{R{r^2_{cl} + h^2}}  
\end{equation}

where $Y$ is in $g$, $b_1 = 0.393$, $b_2 = 0.576$, $b_3 = -0.107$, $b_5 = -0.899$, $b_V = -0.200$, $V_A = 1112$, $h(km) = 6.91$, $\sigma_{ln\ Y} = 0.612$.



