
\setlength{\paperwidth}{210mm}
\setlength{\paperheight}{297mm}% fixed.

\documentclass{article}

\usepackage[review]{myreviewpckg}

\usepackage{textcomp}
\usepackage{simplemargins}

\clubpenalty=10000  % Orphan - First of paragraph left behind
\widowpenalty=10000 % Widow  - Last of paragraph sent ahead

\setallmargins{1in}

\begin{document}

\setlength{\parindent}{0ex}

January 26, 2017
\vspace{12ex}

Dr.~Mariano Garc\'{i}a Fern\'{a}ndez\\
Editor in Chief, Journal of Seismology\\
MNCN-CSIC, Madrid, Spain
\vspace{8ex}

\setlength{\parskip}{2ex}

Dear Dr.~Garc\'{i}a Fern\'{a}ndez,

We write in reference to your communication of 12 December 2016 regarding our manuscript submission to the Journal of Seismology, No.~JOSE-D-16-00092R1. We want to thank you for your positive reaction to the revised manuscript and your decision to move forward with the publication of our article, upon the suggested minor changes of language, which we are now pleased to inform we have completed. Enclosed you will find the new version of our submission, which we would appreciate if you could approve at your earliest convenience, so we can proceed with the publication process.

This being said, we did not want to let this opportunity pass by without also addressing the points raised by the third reviewer. We were a little intrigued by the strict criticism of this third reviewer, but at the same time understood the motivation behind his/her comments. While we appreciated very much your candid reaction this this review---given our response to the initial evaluation---, we thought that in addition to following your advise about how to handle it (through minor modifications of form and language), it was still appropriate for us to share with you our views about some of these criticisms. You will also find enclosed our responses, which we hope you will appreciate.

We thank you again for your help during the review process, and look forward to working with the editorial team in the final production and publication of the article.

\vspace{5ex}
Sincerely yours,
\vspace{10ex}

Ricardo Taborda\\
Department of Civil Engineering, and\\
Center for Earthquake Research and Information\\
3890 Central Ave., Memphis, TN 38152\\
phone: 901-678-1527; fax: 901-678-4734\\
University of Memphis

% ******************************************************************
% *************************** REVIEWER 3 ***************************
% ******************************************************************

\newpage

\setlength{\parskip}{0ex}
% \setlength{\parindent}{0ex}

\begin{center}
	\bf
	\large
	Authors' Response to Reviewer 3 Comments
\end{center}

\comment{1}{%
As shown in Figure 2 of this manuscript, the major faults and its characteristicparameter uncertainties should be incorporated into the seismic hazard study of the this region. As the authors stated in Section 4, this type of work has been performed in Tavakoli and Ghafory-Ashtiany (1999) and many others. In order to improve the previous studies, therefore, the seismic hazard maps presented in this manuscript should be modified based on the known major fault sources and areal seismic sources, which are considered as background seismic sources. The smoothed seismicity approach could be used for the background seismic source zones. Otherwise, only using the smoothed seismicity approach could lead to misleading of seismic hazard values and hazard maps in this region, especially when the time span of earthquake catalog is not enough to give us the true seismicity pattern of the region. Lack of consideration for major faults, will have significant and far-reaching consequences for Iran and major metropolitan areas such as Tehran. Accepting the publication in a reputable journal such as Journal of Seismology without consideration for major faults will be a mistake. The reviewer is puzzled why the fourth author of this article who is a reputable researcher has agreed on the publication of this article. The authors write: ``We find our results to be helpful in understanding seismic hazard for northern Iran, but recognize that additional efforts are necessary to obtain more robust estimates at specific areas of interest and different site conditions.''
}

\response{%
We acknowledge the fact that it would be preferable for any final comprehensive study of the seismic hazard of a region to consider all forms of seismic sources. We also understand that ideally, these sources are to be characterized in terms of the geologic, tectonic, and seismic knowledge of the region. We, however, are also aware that even when these factors are believed to be known, that might still not be sufficient (see Stein, 2012). The smoothed seismicity method is an accepted approach proven to be useful for building a more complete picture of the seismic hazard of a region, but we do not claim, nor do other authors, that the method is robust or comprehensive enough to address seismic hazard in a definitive manner. The method carries advantages as well as disadvantages, which we believe are well documented and acknowledged in our manuscript. Moreover, we are careful not to recommend our findings as final. We are also careful to mention that seismic design or seismic risk mitigation decision cannot solely based on our findings but on a sum of different studies, some of which may use other additional methods and data we did not use. That is why we ``recognize that additional efforts are necessary to obtain a more robust'' hazard estimation. We, nonetheless, also point out that our findings are in good agreement with other fault-based models. For the city of Tehran, for instance, our results are in the mean range of other studies. We use this as an indication that our results are credible and thus could be considered by others for the background seismicity portion of a more comprehensive study.
}

\comment{2}{%
The attenuation relationship used in this study is based on the Joyner-Boore distance, while it appears that the epicentral distance has been used in the smoothed seismicity approach. The authors have not considered the distance adjustment factors, and this lack of distance conversion could lead to underestimate the hazard values in the region, in particular, at low probability of exceedance [2\% PE in 50 year].
}

\response{%
In the smoothed seismicity approach each earthquake as assumed as a point source. That being the case, for any given earthquake in the catalog, the dimensions of the rupturing portion of the fault (length and width) are reduced to a single point in space; and the rupture size and rupture direction become irrelevant. This is not our choice but a fundamental premise of the smoothed seismicity method. The Joyner-Boore distance, on the other hand, is defined as the closest horizontal distance to the vertical projection of the rupture. In finite fault models, where both fault length and rupture direction are meaningful aspects of the source representation, appropriate adjustments need to be done to compute the Joyner-Boore distance with respect to epicentral distance, but that is not the case here. In other words, for point sources, as considered in the smoothed seismicity approach followed, the Joyner-Boore distance is the same as the epicentral distance. It is therefore valid to use the chosen attenuation relationship with epicentral distances.
}

\comment{3}{%
In Figure 4 of this manuscript, the authors have only added the earthquake magnitudes of 3 and greater into a catalog with minimum magnitude of 4 and greater since the year of 2000, and have claimed that the magnitude level of 3--4 is complete from the year of 2005 for all area sources. The authors should explain why they haven't completed their catalog for magnitude 3 and greater for all time. This might affect the period of completeness, declustering and number of events for the recurrence rates.
}

\response{%
We did not consider earthquake with magnitude 3 and above before 2000, simply because there is no data available before that date in a formal catalog. Instrumentation and IIEES reports for earthquakes in the magnitude level 3--4 are well documented only after 2000. The recent study by Zare et al.~(2014) reported the construction of a catalog for all time for a minimum magnitude of 4. We took upon the task of augmenting this catalog by adding the more recent earthquakes after 2012 and the weaker events in the magnitude level 3--4 since 2000. Our completeness analysis of the data shows that the earthquake with magnitude 3 and above is complete from 2005, and given the recurrence rate of smaller earthquakes, we believe the data collected between 2005 and 2015 constitutes a reliable sample for the analysis. We should also note that both softwares used in this study (i.e., HA3 developed by Dr. Kijko, and Seiskit developed by Dr.Frankel) can handle this discontinuity in the catalog.
}

\comment{4}{%
There are few ground motion prediction equations (GMPEs) developed for the Iranian plateau such as Nowroozi (2005), Ghasemi et al.~(2009), Saffari et al.~(2012), and Kale et al.~(2015). The authors do not use these models, and use the model by Kalkan and Gulkan (2004), which has been developed for Turkey. The authors provides a lengthy argument why they do not follow the standard procedure of considering credible GMPEs for the study site. It is not clear why the authors use $h = 6.91$~km and $V_A = 1,112$ m/s. There are no discussions on what the authors consider as their reference site for the PSHA analysis. Is this model developed for a reference site of 1,112 m/s? If so, why?
}

\response{%
The reviewer stands correct when saying that there are other GMPEs that have been developed for the Iranian plateau, different to the one used in our study. Most of these GMPEs, however, are adjusted to fit a certain dataset, as opposed to a complete catalog from the whole northern Iran region. As explained in our manuscript, Zafarani and Mousavi (2014) conducted a careful study in which several local, regional, and next generation attenuation (NGA) GMPEs were evaluated with the objective of determining which GMPEs fitted the data of northern Iran best. Out of the nine GMPEs considered by Zafarani and Mousavi (2014), the one developed by Kalkan and Gulkan (2004) showed the best fits for peak ground acceleration at period, $T = 0$~s. Our hazard analysis is based precisely on PGA at $T=0$~s, for rock sites. Although other GMPEs are more appropriate for other intensity measures and/or different periods, we limited our analysis to this particular intensity measure and thus our selection. We should note, for instance, that the GMPE of Ghasemi et al.~(2009) referenced by the reviewer was one of those evaluated by Zafarani and Mousavi (2014), but unfortunately this GMPE does not include an attenuation relationship for PGA. Other authors mistakenly use spectral accelerations at $T = 0.05$~s as estimation for PGA, but we preferred not to do so. It is also true that the GMPE of Kalkan and Gulkan (2004) was originally developed using data from Turkey, but that does not mean its applicability should be limited to Turkey alone. Several studies have shown before that GMPEs from specific regions can be used in other places in the world if shown to fit local GMPEs or data well (e.g., Campbell and Bozorgnia, 2006, 2015), and this is a common practice in hazard and risk analysis for regions with limited data or attenuation relationships. Last, the reason why we do not change the parameters $h$, $V_A$ or $V_S$ is because doing so would alter the fit of the GMPE with the data.
}

\comment{5}{%
The model provided is slightly different than a recently published work by Khodaverdian et al.~(2016) with some slightly different considerations for various models and parameters, which are not really physics-based information. The authors write: ``Admittedly, these are subjective decisions and any reasonable choice can be considered valid or within the margins of the associated epistemic uncertainty.'' The authors need to provide solid evidence why their model advances Khodavaerdian et al.~(2016).
}

\response{%
We started to work on this paper in the summer of 2015 and submitted the first version of the manuscript for review in the summer of 2016. By pure coincidence, and without our knowledge, Khodaverdian et al.~submitted the first version of their work to the Bulletin of the Seismological Society of America in the summer of 2015, and their work was published in the summer of 2016. We had no way of knowing about their work before it was published, but once we learned about their publication we took the time to delay our submission a couple of months to modify our manuscript so it could reflect on the differences between their work and ours. As summarized in the manuscript, the main difference in the application of the smoothed seismicity method is the treatment of the $b$-value and $M_{\mathrm{c}}$, which they computed and smoothed within a larger grid than the one used for $a$ values. We argue in the manuscript that their approach may lead to over-smoothing across seismic regions and this is, in our view, not the best alternative. We, however, as stated in the quote highlighted by the reviewer, admit that this is a subjective decision which can be considered within the margins of the epistemic uncertainty intrinsic to studies of this nature. We also highlight the fact that the study of Khodaverdian et al.~focuses primarily in the determination of the seismic parameters, while we take the analysis to the hazard level, not only for a single model but for a combination of models that account for uncertainty. In addition, we also note that we used a more complete catalog, which includes recent strong earthquakes in the 2012--2015 period not considered by Khodaverdian et al. That said, we do not pretend to claim we intended to advance Khodaverdian et al., we simply happened to be working on a similar subject around the same time, with some differences which warrant, in our view, the publication of alternative results.
}

\end{document}
