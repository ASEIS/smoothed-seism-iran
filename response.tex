
\setlength{\paperwidth}{210mm}
\setlength{\paperheight}{297mm}% fixed.

\documentclass{article}

\usepackage[review]{myreviewpckg}
\usepackage{simplemargins}

\clubpenalty=10000  % Orphan - First of paragraph left behind
\widowpenalty=10000 % Widow  - Last of paragraph sent ahead

\setallmargins{1in}

\begin{document}

\begin{center}
	\bf
	\large
	Authors' Response to Reviewer 1 
\end{center}

\noindent
We thank the reviewer for his/her comments and suggestions, which helped us improve the manuscript. To help the review process, in the annotated manuscript version of the paper, we have highlighted the changes that resulted from these comments in green colored font when appropriate. In the following, we provide a response to each comment. The original comments are in italic black font, followed by our responses in regular blue font.
\vspace{2ex}
\newline 

\introcomment{~}{%
The authors perform a seismic hazard analysis for northern Iran using a newly developed source model from smoothed seismicity. The work is an important contribution to seismic hazard analyses in Iran. Development of the seismic source model follows a careful analysis of the earthquake catalogs for information about spatially varying earthquake rates, maximum magnitudes and b-values. The authors are careful to acknowledge that alternative means for estimating earthquake recurrence - and for other components of the source and ground motion characterization - also exist, and that this work is only one approach for characterizing hazard in the region.
}

\comment{1}{%
Pp3, Line 71: A short, summary description of previous source models would highlight the contribution of this work. In particular, summary details of the work by Khodaverdian et al should be included to indicate how the approach of the authors differs from previous work. 
}

\response{%
Response will be here.
}

\comment{2}{%
Pp 3, line 77: Fault-based/geologic-based earthquake recurrence estimates 
}

\response{%
Response will be here.
}

\comment{3}{%
Pp 4, line 106: Should this be northwest?
}

\response{%
Fixed, Thanks.
}

\comment{4}{%
Pp 6, line 168: It would be good to add a reference for the GEM effort.
}

\response{%
Response will be here.
}

\comment{5}{%
Pp7-8: Discussion about identifying the completeness level of the earthquake catalogs is a bit confusing. Typically, one either specifies a minimum magnitude of completeness (Mc) and then tries to identify the earliest time for which events or complete; or, one specifies a year and computes the corresponding Mc. In reading the text, it appears that both methods were used. Please clarify.
}

\response{%
You are right. We used both methods in the study. First we compute the $b-value$ and $M_{max}$ for each region, or the uniform model using the HA3 package written by Dr. Kijko. $M_c$ is used as an input in the software to get accurate results. We computed the smoothed seismic hazard using the Seiskit package, written by Dr. Frankel. In estimating the seismic hazard, specifically $a-value$, we need to define a start year of completeness for each magnitude range. We modified the text to clarify the statement. 
}

\comment{6}{%
Pp7-8: The authors should note that some authors use an alternative smoothing kernel $(~exp(-(D/2*sigma)^2)$. Comparison of the smoothing length between different studies may require consideration of a $sqrt(2)$ factor.
}

\response{%
Added. Thanks.
}


\comment{7}{%
Pp7-8: In addition to the Mirzaei et al. (1997) reference, some authors employ likelihood testing of smoothed seismicity rate models to identify optimal smoothing lengths for adaptive and fixed smoothed seismicity methods (e.g., Werner et al., 2011; Moschetti, 2015). 
}

\response{%
Added to the text. Thanks.
}

\comment{7}{%
Pp 9: Kagan and Jackson, 1994 should also be included in the reference list.
}

\response{%
Added. Thanks.
}

\comment{8}{%
Pp 10, line 302: " ... subjective decisions and any reasonable choice ...  " I would disagree that these are subjective decisions. My opinion is that you have applied scientific judgment. You clearly need to communicate where judgment is applied; however, \\
- Details about the implementation of the two seismicity rate models (U, R) should be expanded. Are sub-catalogs developed for the five separate regions and then used to develop separate a-grids? Presumably, the separate rate files are then used to compute hazard curves and are summed at each location. \\
-A logic tree would be of great benefit in clarifying how the source models are combined. In particular, it would clarify the weighting from different completeness levels.\\
-Plots of seismicity rates from the R and U models would be of interest and particularly relevant given that the smoothed seismicity model is the main contribution of this work to hazard analyses for Iran.
}

\response{%
We added the logic tree to the study to consider the epistemic uncertainty, instead of giving the range of uncertainty. In practice, we define the grids for the whole region, however, estimate the $a-value$ and also the probability of exceedance for each region, separately. To do that, in each run we just provide the seismic source data ( in this case location and magnitude) for the specific region. Based on the logic tree approach we conducted 162 runs to get the results $(3 M * 3 B * 6 R(5 region+1 uniform)*3 GMPE)$.
}

\comment{9}{%
Pp 14, line 391 The incorporation of smaller earthquakes $(M<5)$ will have a significant influence on the hazard associated with PGA. Recent papers by Atkinson and others have demonstrated this effect. You may wish to argue that Mmin was chosen based on the expected smallest magnitude capable of causing damage.
}

\response{%
Thanks for the comment. We computed hazard for events $M_w > 4.5$ and modified the text. 
}

\comment{10}{%
Figure 7: Is the magnitude-frequency distribution just a truncated (Mmin, Mmax) GR model? Figure 7 suggests that the MFD may be tapered?
}

\response{%
Yes it is truncated GR model. (Figure caption is not fixed yet.)
}

\comment{11}{%
Pp 12, line 345: "... controls the spatial extent to which the local occurrence of earthquakes ..." I would argue that the smoothed seismicity controls the local occurrence and spatial distribution of earthquakes and that the GMPEs control the intensity level at a particular site given earthquake locations and magnitudes from your source model.
}

\response{%
Thanks. Modified it. 
}

\comment{12}{%
Pp 13, line 361: Change "approached" to "approach" and "lead" to "led."
}

\response{%
Done.
}

\comment{13}{%
The approach to generating Figure 10 should be clarified. It appears that the probabilistic ground motions from the R and U model are averaged and used to compute a standard deviation. The formal method for combining alternative models in PSHA is through the logic tree; this is equivalent to weighting the models in hazard (rate) space, rather than in ground motion (intensity, PGA) space. It appears that the authors have done the latter, though I may be incorrect. If this is the case, please correct. It is very reasonable to combine the R and U models to produce another alternative model. This work is not a comprehensive seismic hazard analysis, and presentation of probabilistic ground motions from the R, U and a weighted model would be reasonable goal. However, I would caution against comparing probabilistic ground motion maps with maps of seismicity because the varying b- and Mmax-values for the R model greatly complicate the comparison. As mentioned earlier, formal testing methods exist for ensuring consistency of observed earthquake locations with the smoothed seismicity models (see CSEP references, Zechar, Schorlemmer, Werner, Helmstetter, Jackson; Moschetti, 2015). Applying these methods is beyond the scope of this work, but might be recognized as a future direction.
}

\response{%
Thanks for the informative comment. In Figure 10, we combined the regional (R) and uniform (U) results (We combine the probability of exceedance, then compute the pga)  with similar weights (i.e., 0.5). The positive and negative standard deviation plots comes from GMPE+Uncertainty  and GMPE-Uncertainty. We were showing the range of variation for PGA around the median value. As we discussed before, in order to consider the epistemic uncertainty, instead of presenting the range of uncertainty, we used the logic tree approach and modified Figure 10. The weights are presented in logic tree figure.
}

\newpage

\begin{center}
	\bf
	\large
	Authors' Response to Reviewer 2 
	\end{center}

\noindent
We thank the reviewer for his comments and suggestions, which helped us improve the manuscript. To help the review process, in the annotated manuscript version of the paper, we have highlighted the changes that resulted from these comments in green colored font when appropriate. In the following, we provide a response to each comment. The original comments are in italic black font, followed by our responses in regular blue font.
\vspace{2ex}
\newline 

\introcomment{~}{%
Review of Seismic Hazard Estimation of Northern Iran Using Smoothed Seismicity by Naeem Khoshnevis, Ricardo Taborda, Shima Azizzadeh-Roodpish, and Chris H. Cramer
}

\introcomment{~}{%
In this paper, the researchers present a new seismic hazard analysis of northern Iran. The paper makes use of the latest available data and methods, is thorough in its review of previous work, and is very well presented and written. I have a few suggestions that I think would improve the impact of their results. The annotated version of the manuscript contains additional and more specific suggestions.
}

\comment{1}{%
The authors present uncertainty in their hazard estimate but do not include a more exhaustive logic tree analysis. Effectively, they have two branches: a regional model and uniform model. It appears that they apply the aleatory uncertainty in the ground motion prediction relation to a form of epistemic uncertainty, effectively double counting this uncertainty in the hazard analysis. I would prefer to see a more standard logic tree analysis where multiple GMPEs (or a backbone plus and minus one standard deviation) and uncertainty in a-value, b-value, and Mmax, for example, are considered. Barring this analysis, I would remove figure 10 and references to hazard uncertainty.
}

\response{Thanks for the comment.  Based on explanation of GMPEs section (please see comment no.?) we decided to define a suite of GMPE as well as plus and minus one standard deviation. We consider the epistemic uncertainty in tectonic seismic regions, $M_{max}$, $b-value$, and attenuation relationship through a standard logic tree analysis.}. 

\comment{2}{%
While I think that the regional and uniform models are generally correct within 1.5 degrees of the border, I think that outside of this region, they may be moderately underestimating hazard by not considering sources with 200 km of their region of interest. Their buffer zone for sources is presently 50 km.
}

\response{That is right. We increased the buffer zone up to 2 degrees. (Haven't modified the text)}

\comment{3}{%
It would be good if the abstract is more specific about the results. Where is the hazard greatest? What sources contribute to regions of higher hazard?
}
\response{Response will be here.}

\comment{4}{%
In the second paragraph of the introduction, I would include a reference to a figure of the region to help readers less familiar with the area. This figure should also be referred to in subsequent paragraphs. It is probably sufficient to move reference to figure 1 here.
}
\response{We  added the reference for figure 1. Thanks.}

\comment{5}{%
line 39:   eight  $\to$ eighth
}
\response{Fixed.}

\comment{6}{%
line 55   focuses in $\to$ focuses on
}
\response{Done. Thanks.}

\comment{7}{%
line 56 $\to$ Please be more specific about how your analysis differs from previous analyses.
}
\response{Response will be here.}

\comment{8}{%
line 60 $\to$ Can you be more specific? What are the other indicators?
}
\response{Response will be here.}

\comment{9}{%
line 71 $\to$ Include reference? Include reference.
}
\response{Done. Thanks.}

\comment{10}{%
line 78 $\to$ reduce $\to$ acknowledge
}
\response{Fixed.}

\comment{11}{%
line 99 $\to$ Is 0.5 degrees enough? To what distance do you do your ground motion calculations? In the western United States, they are done to 200 km. In the central and eastern, they are done to 1000 km. Depending on this value, you should include sources up to this distance from your region of interest.
}
\response{We reanalyze the hazard with 2 degrees buffer zone. (Not modified in the text.)}

\comment{12}{%
line106 $\to$ northeast $\to$ northwest
}
\response{Done. Thanks}

\comment{13}{%
line 114 $\to$ There is no previous mention of this earthquake. I am not sure to which earthquake you are referring. Should either the 1721 or 1780 earthquakes be the Shebli earthquake?
}
\response{Thanks for the comment. It is 26 April 1721,$M_s ~ 7.3$ earthquake}

\comment{14}{%
line 180 $\to$ The smaller events should also help to inform the G-R relationship.
}
\response{Thanks for the comment. Added to the text.}

\comment{15}{%
line 182 $\to$ How come this study doesn't appear in Table 3. Do they not calculate PGA for 2 and 10\% exceedance in 50 years?
}
\response{They didn't provide a numerical value, or numbers on the contour lines, however, we read the images with Matlab and locate the values. We added to the table. Thanks.}

\comment{16}{%
line 192 $\to$ What is the total number of events before and after declustering?
}
\response{Number of events before decluttering: 10441, total foreshock: 971, total aftershock: 4206, declustered data 5254. It should be noted that, these numbers are representing the events in the study region (including the bufferzone). It partly includes events from Central and Zagros region. For completeness study we studied the whole data for each region.}

\comment{17}{%
line 224 $\to$ Why did you use this method? What are the benefits of one or the other?
}
\response{Response will be here.}

\comment{18}{%
line 232 $\to$ It is unclear how you chose Mc. In some cases, it looks like you chose the peak goodness-of-fit value, but in others you did not. Please be more specific about how Mc was chosen.
}
\response{According to Wiemer and Wyss (2000), the priority is picking the magnitude which has the peak goodness-of-fit value. However at some cases for two or a couple of magnitudes the goodness-of-fit scores are not considerably different. In these cases, we pick the minimum magnitude, even though it doesn't have the peak goodness-of-fit value, in order to preserve the valuable earthquake event data.}

\comment{19}{%
line 267 $\to$ add 'and'
}
\response{Added. Thanks.}

\comment{20}{%
line 289 $\to$ You might also cite the 2014 USGS National Model (Petersen et al., 2014) and the Afghanistan seismic hazard maps (Boyd et al, 2007), which include spatial variability in the smoothing parameter.
}
\response{Added. Thanks.}

\comment{21}{%
line 297 $\to$ provided below $\to$ provided in the next section
}
\response{Fixed.}

\comment{22}{%
line 300 :  allows to draw information $\to$ allows one to draw information 
}
\response{Done.}

\comment{23}{%
line 323 $\to$ Be more specific about how HA3 computes b and Mmax. What is its method?
}
\response{Method added. Thanks.}

\comment{24}{%
line 335 $\to$ Different seismic zone will not necessarily have difference values of b and Mmax. Change to this: Different seismic zone may have different values of b and Mmax.
}
\response{Modified. Thanks.}

\comment{25}{%
line 337 $\to$ a means to reduce epistemic $\to$ a means to account for epistemic 
}
\response{Done.}

\comment{26}{%
line 361 $\to$ lead $\to$ leads
}
\response{Two reviewer suggested two different word. Both of them are correct. Please see.}

\comment{27}{%
line 364 $\to$ How do you address the epistemic uncertainty generally introduced by using a suite of GMPEs?
}
\response{As mentioned in Atkinson and Adam (2013) and Atkinson et al (2014), using different GMPEs with different logic tree weights from different literature is not necessarily the best way to model the epistemic uncertainty in GMPEs. We prefer to use the alternative GMPEs and applicable data to guide the choice of a representative or ?central? GMPE, and to define representative (upper and lower) GMPEs that express uncertainty about the central GMPE. In the previeous study (Khodaverdian 2016), they used Akkar and Bommer (2010) and Ghasemi et al (2009) along side other equations. Akkar and Bommer (2010) derived the equation up to 100 Km, which they used for 200 km. Also Ghasemi et al (2009) is presented the coefficients for spectral acceleration not PGA. Of course this alternative method has been a hotly-debated topic, and there is no clear consensus about that, however, the plots of attenuation relationships in different magnitude illustrate the fact that with this approach we can address the epistemic uncertainty in an acceptable range. We totally agree with the reviewer?s comment that presenting the seismic hazard with upper and lower values (Fig 10) to address the epistemic uncertainty could be misleading for the readers. In order to address these concerns, we do the seismic hazard analysis for bigger domain (2 degrees buffer zone) and combine the results according to the logic tree scheme in Fig X.}

\comment{28}{%
line 365: add 'and'
}
\response{Added. Thanks.}

\comment{29}{%
line 370 $\to$ Could you consider other depths to help fill out your range of epistemic uncertainty?
}
\response{The Kalkan and Gulkan (2004) attenuation relationship is originally generated for Turkey. However, in the recent test by Zafarani and Mousavi (2014), it gives acceptable results for PGA. Since the high accuracy achieved by the current depth, and we have no idea how different depth will change the results, we prefer to not to use a parameter without knowing the effects. }

\comment{30}{%
line 371 $\to$ How does this value compare with the models you decided not to use?
}
\response{Response will be here.}

\comment{31}{%
line 376 $\to$ In this case, you should consider all sources with 200 km of your area of interest.
}
\response{We modified the buffer zone. Thanks.}

\comment{32}{%
line 394:  50-years $\to$ 50-year
}
\response{Done.}

\comment{33}{%
line 420: with concentration $\to$ with a concentration
}
\response{Added. Thanks.}

\comment{34}{%
line 422 $\to$ I would have stated this differently. It looks like the lower b-value in Kopeh Dagh yields relatively high hazard in the R model whereas in the U model, this low b-value has leaked into the entire region and subsequently elevated seismic hazard in all areas outside of Kopeh Dagh.
}
\response{Response will be here.}

\comment{35}{%
line 428 $\to$ Why equal weights?
}
\response{We added a comprehensive logic tree analysis after revision and modified the weights. Please see the text.}

\comment{36}{%
line 431 $\to$ I'm not sure I quite understand this. The standard deviation in the attenuation relation is considered aleatory uncertainty and is present within the hazard curve. Adding it as epistemic as well will cause double counting. If you want to include figure 10, a proper logic tree analysis to include more forms of epistemic uncertainty should be performed. For example, add branches for uncertainty in which GMPE you choose, uncertainty in b-value, uncertainty in Mmax, uncertainty in a-value, uncertainty in smoothing parameter, etc.
}
\response{We modified the approach and considered the mentioned uncertainties.}

\comment{37}{%
line 434 $\to$ Just because the basic pattern appears to better follow faults in the region does not mean it's a better assessment, especially if the b-values are truly different between regions and better represented by the regional analysis.
}
\response{Response will be here.}

\comment{38}{%
line 439: of city of Tabriz $\to$ of the city of Tabriz.
}
\response{Fixed. Thanks.}

\comment{39}{%
line 450 $\to$  for 2 and 10\%, respectively $\to$ for 2 and 10\% in 50 years, respectively.
}
\response{Done.}

\comment{40}{%
line 453 $\to$  for a 10\% probability of exceedance ?  for a 10\% probability of exceedance in 50 years.
}
\response{Done.}

\comment{41}{%
line 455 $\to$ 0.27-0.31 g for 10\% probability of exceedance $\to$  0.27-0.31 g for the same exceedance probability
}
\response{Done.}

\comment{42}{%
line 458 $\to$  for 2 and 10\% ? for 2 and 10\% in 50 years
}
\response{Done.}

\comment{43}{%
line 461: our results are close or within $\to$ our results are close to or within 
}
\response{Done.}


\comment{44}{%
line 462 $\to$ Can you explain the differences in their analysis relative to yours that lead to higher values?
}
\response{Response will be here.}

\comment{45}{%
line 466 $\to$ exceedance as functions of PGA ? exceedance as functions of PGA, also known as hazard curves}
\response{Added. Thanks.}

\comment{46}{%
line 475 $\to$ Can you explain the differences in the hazard analysis that led to differences in hazard?}
\response{Response will be here.}

\comment{47}{%
line 462 $\to$ Can you explain the differences in their analysis relative to yours that lead to higher values?
}
\response{Response will be here.}

\comment{48}{%
lline 488 $\to$ Not sure you can say this because you don't include fault specific sources, and you do not consider a more full range of epistemic uncertainty, e.g. you only use one GMPE (though you do consider two b-value and Mmax based models).
}
\response{Response will be here.}

\comment{49}{%
line 488:  hazard analysis $\to$ hazard analyses
}
\response{Done.}

\comment{50}{%
line 490: models as a mean $\to$ models as a means}
\response{Done.}

\comment{51}{%
line 493 $\to$ Though you suggest the uniform model is unrealistic on line 334.}
\response{Response will be here.}

\comment{52}{%
line 502 $\to$ for providing advise ? for providing advice}
\response{Done.}

\comment{53}{%
line 509 $\to$  Should this be Sunset?}
\response{Yes. Thanks.}

\comment{54}{%
p 25, table2 $\to$ What did you do for the earthquakes occurring outside of these five zones?}
\response{There are earthquakes out of the regions, mainly above the Caspian see. Since those area are not in our region of interest (They mainly are in country of Azerbaijan), and are fairly far from our region, we ignored them.}

\comment{55}{%
p 26, table3 $\to$ Specify units.}
\response{Done. Thanks.}

\comment{56}{%
p32, figure 3 $\to$ You should also show events in the buffer zone used to compute hazard.}
\response{Done.}

\comment{57}{%
p34, figure 5 $\to$ Which seismic zone corresponds to which color? It looks like either reporting or rates have changed over time. For example, I might argue that the red region was not complete for M3 and above until 2005. }
\response{Thanks for the comment. The problem with that region is having less than and equal sign on magnitude 4. There are many earthquakes go to that category because of that condition. Since, converting earthquake magnitude always has an error margin, we think the comment is relevant. Based on the data we assume 2005 as a completeness magnitude. We presented the legends at the bottom of the figure.}

\comment{58}{%
P37: 10 percent of probability $\to$ 10 percent probability }
\response{Response will be here.}

\comment{59}{%
p 39,figure 11: Include horizontal lines indicating 2 and 10\% probability of exceedance in 50 years.}
\response{Included.}

\comment{60}{%
p 41, figure 12: Can you include curves or points for the other studies in Table 3?}
\response{Added.}


\introcomment{~}{%
Summary of modifications (May or may not addressed as comment.)\\
- Buffer zone updated.\\
- Data updated.\\
- b-values updated.\\
- Completeness parameters updated.\\
- Figure 1: Buffer Zone Updated.\\
- Figure 3: Seismicity.sh modified. Historical.txt and Instrumental.txt updated to new data. \\
- Figure 4: Updated.\\
- Figure 5: The excel file is updated. The figure needs to be updated.\\
- Figure 6: Updated data completeness plots added to the figure. Figure needs final polishing. \\
- Figure 7: Updated.\\
- Figure 11: Updated.\\ 
- Figure 8: Updated.\\
- Figure 10: Updated.\\
}

\end{document}
