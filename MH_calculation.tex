\section{Methodology for Hazard calculation}

To calculate the seismic hazard, we use spatially smoothed seismic hazard analysis \citep{Frankel1995}. In this model, seismic events are spatially gridded to cells. We are attempting to assess the relative likelihood of moderate earthquakes ($M_w > 4.5$), which cause structural damage. According to \citet{Frankel1995}, moderate earthquakes generally occur in areas that there have been significant numbers of events of magnitude 3 and above. Therefore, these events provide a reasonable guide to where moderate earthquakes will most likely occur. Since the catalog's completeness for $M3+$ and $M4+$ is different, we use two magnitude completeness range for less than $M_w5$ . Category of 5+ and others (M6+ and M7+) assume that future $M_w4.5$ events will occur near where they have occurred in the past. \\
\noindent
According to \citep{BHRC2014}, $M_w5$ is considered as a threshold magnitude to structural damage. In the smoothed seismicity method, the model uses each event location as a point source, hence $M_w4.5$ could be damaging earthquake if it occurs very close to the structure. In this study, in order to consider the probabilistic seismic hazard of both models, we defined two models based on $M_w5$ and $M_w4.5$ as a threshold magnitude for seismic hazard calculation with weight of 0.7 and 0.3, respectively.\\
\noindent
According to \citet{Frankel1995}, we count the number of earthquake $n_i$ with magnitude above $M{_R{_e{_f}}}$ in each cell of a grid with spacing of $0.1^{\circ}$ in both latitude and longitude (about 11 km). This count shows the maximum likelihood estimate of $10^a$ for that mentioned cell \citep{Weichert1980, Bender1983} for earthquakes with magnitude greater than $M{_R{_e{_f}}}$. Applying the Herrmann formula \citep{Herrmann1977}, the values of $n_i$ are converted from cumulative values (number of events above $M{_R{_e{_f}}}$) to incremental values (number of events from $M{_R{_e{_f}}}$ to $M{_R{_e{_f}}} +\Delta M$ ). The grid of $n_i$ values is spatially smoothed by multiplying by a Gaussian function with correlation distance c. For each cell $i$, the smoothed value $\tilde{n}_i$ is obtained from \citet{Frankel1995}


\begin{equation}
\tilde{n_i}=\frac{\sum_{j} n_{j} e^{\frac{-\Delta_{ij}^{2}}{c^2}}}{\sum_{j} e^{\frac{-\Delta_{ij}^{2}}{c^2}}},
\end{equation}

\noindent
where,  $\tilde{n}_i$  is normalized to preserve the total number of events. $\Delta{_i{_j}}$ is the distance between the $i{_t{_h}}$ and $j{_t{_h}}$ cell. The sum is calculated over cells $j$ within a distance of $3c$ of cell $i$.\\
\noindent
For a grid of sites and by using $\tilde{n}_i$ from Eq. (1), the annual probability of exceeding specified ground motions is calculated. For each site, the values of $\tilde{n}_i$ are binned by their distance from that site, in a way that $N_k$ denotes the total of $\tilde{n}_i$ values for cells within a certain distance increment of the site. Now the annual rate $\lambda (u> u_0 )$ of exceeding ground motion $u_0$ at a specific site is determined from a sum over distance and magnitude \citet{Frankel1995}:



\begin{equation}
\lambda(u>u_{0}) = \sum_{k}\sum_{l}10^{[log(\frac{N_{k}}{T}-b(M_l-M{_r{_e{_f}}}))]} p(u>u_0 | D_k ,M_l),
\end{equation}

\noindent
where $k$ is the index for the distance bin, and $l$ is the index for the magnitude bin; $T$ is the time in years of the earthquake catalog used to determine $N_k$. Annual rate of earthquakes in the distance bin $k$ and magnitude bin $l$ is the first factor in the summation. The $b-value$ is considered uniform throughout most of the area. $P(u>u_0 | D_k,M_l )$ is the probability that for an earthquake at distance $D_k$ with magnitude $M_l$, u at the site will exceed  ground motion $u_0$. This probability depends on the attenuation relation and the standard deviation (aleatory variability) of the ground motion for any specific distance and magnitude \citep{Frankel1995}


