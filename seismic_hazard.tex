\section{Seismic Hazard}

Occurrence of devastating historical earthquake and also essential needs for constructing new buildings and infrastructures inspired researchers to study the seismic hazard of northern Iran from different points of view. Dividing Iran into 20 tectonic seismic regions, \citet{Tavakoli1999} prepared iso-acceleration contour lines and seismic hazard zonation for return periods of 75 and 475 years corresponding to 50\% and 10\% probability of exceedance in 50 years, based on major known lines and area source models. According to their study, North Tabriz fault zone and north of Tehran fault zone have the highest acceleration. Maximum mean acceleration in the vicinity of these tectonic elements is predicted to be around 0.45 $g$ and 0.3 $g$ for a return period of 475 and 75 years, respectively. In the smaller scale, using the logic tree approach to compensate for uncertainties in the attenuation relationship, \citet{Ghodrati2003} presented a probabilistic seismic hazard assessment of metropolitan Tehran, the capital of Iran. The results showed that the PGA ranges from 0.27-0.46 $g$ for a return period of 475 years and from 0.33-0.55 $g$ for a return period of 950 years. 
In this study we use background seismicity to estimate the seismic hazard in northern Iran. Using background seismicity, \citet{Cao1996} conducted a seismic hazard estimation study in southern California. They concluded that one could obtain similar results implementing background seismicity to those obtained by introducing zones for areas with well-recorded seismicity patterns. \citet{Lapajne1997} used spatially smoothed seismicity modeling to acquire a seismic hazard outlook in Slovenia. They defined 4 different models with different correlation distances, including a model based of the total released seismic energy. They gained an acceptable PGA in comparison with other seismic hazard approaches. \citet{Akinci2004} estimate the seismic hazard in central and northern Italy using smoothed historical seismicity. Their results showed that the smoothed seismicity approach gives reasonable regionalized results without introducing seismogenic zones. This approach has also been used in many regions with different seismicity patterns \citep{Wesson1999, Klein2001, Hamdache2008, Kalkan2009, Moschetti2014, Boyd2008}.

\subsection{Analysis Method}
\subsubsection{smoothed seismicity method}
We use spatially smoothed seismic hazard analysis \citep{Frankel1995} to estimate the seismic hazard. In this model, seismic events are spatially gridded to cells. We are attempting to assess the relative likelihood of moderate earthquakes ($M_w > 4.5$), which cause structural damage. According to \citet{Frankel1995}, moderate earthquakes generally occur in areas that there have been significant numbers of events of magnitude 3 and above. Therefore, these events provide a reasonable guide to where moderate earthquakes will most likely occur. Since the catalog's completeness for $M_w > 3$ and $M_w > 4$ are different, we use two magnitude completeness range for less than $M_w 5$ . Earthquake greater than magnitude 5, most probably will occur near where they have occurred in the past. \\
\noindent
According to Building and Housing Research Center  \citep{BHRC2014}, $M_w 5$ is considered as a threshold magnitude for onsetting the structural damage. In the smoothed seismicity method, the model uses each event location as a point source, hence $M_w 4.5$ could be damaging earthquake for old masonry buildings or even engineering building if it is very close to it, in this study, in order to consider the probabilistic seismic hazard, we defined two models based on $M_w 5$ and $M_w 4.5$ as a threshold magnitude for seismic hazard calculation. \\
\noindent

In order to calculate the annual rate ($\lambda$) of exceeding ground motion ($u$) at a specific site, according to \citet{Frankel1995}, first we need to divide the region into cells (0.1*0.1) and count all earthquake events which are bigger than $M_{ref}$ in each cell. We use Herrmann formula \citep{Herrmann1977} to convert the cumulative (i.e. number of events bigger than $M_{ref}$) values to incremental (i.e. number of events from$M_{ref}$ to $M_{ref}+\Delta_m$) values. Using Gaussian function we smooth the number of events in each cell. The smoothed value $\tilde{n}_i$ is obtained from

\begin{equation}
\tilde{n_i}=\frac{\sum_{j} n_{j} e^{\frac{-\Delta_{ij}^{2}}{c^2}}}{\sum_{j} e^{\frac{-\Delta_{ij}^{2}}{c^2}}},
\end{equation}

\noindent
where,  $\tilde{n}_i$  is normalized to preserve the total number of events. $\Delta{_i{_j}}$ is the distance between the $i{_t{_h}}$ and $j{_t{_h}}$ cell. The sum is calculated over cells $j$ within a distance of $3c$ of cell $i$.\\
\noindent
Then we discretized the magnitude and distance in uniform bins and get the total number of Normalized, smoothed events ($N_k$) inside the certain distance increment.
Having ($N_k$) and $T$ (catalog time duration that we calculated $N_k$ within that duration), we can calculate the annual rate  $\lambda (u> u_0 )$ of exceeding ground motion according to 

\begin{equation}
\begin{split}
\lambda(u>u_{0}) = \sum_{k}\sum_{l}10^{[log(\frac{N_{k}}{T}-b(M_l-M{_r{_e{_f}}}))]} \\
p(u>u_0 | D_k ,M_l),
\end{split}
\end{equation}

\noindent
where $k$ and $l$ are denoted as index of distance and magnitude bin. $P(u>u_0 | D_k,M_l )$ is the probability that for an earthquake at distance $D_k$ with magnitude $M_l$, u at the site will exceed  ground motion $u_0$. This probability depends on the attenuation relation and the standard deviation (aleatory variability) of the ground motion for any specific distance and magnitude \citep{Frankel1995}.








% -----------------------------------------------------------------------
% -----------------------------------------------------------------------

%%% My (Naeem) original text


%\subsection{Background}
%
%Regarding historical earthquakes and the need for constructing new buildings and infrastructure, different seismic hazard analyses were conducted in Iran. Dividing Iran into 20 tectonic seismic regions, \citet{Tavakoli1999} prepared iso-acceleration contour lines and seismic hazard zonation for return periods of 75 and 475 years or 50\% and 10\% probability of exceedance in 50 years, based on major known lines and area source models. According to their study, North Tabriz fault zone and north of Tehran fault zone are in the highest acceleration contour. Maximum mean acceleration in the vicinity of these tectonic elements is predicted to be around 0.45 $g$ and 0.3 $g$ for a return period of 475 and 75 years, respectively. In the smaller scale, using the logic tree approach to compensate for uncertainties in the attenuation relationship, \citet{Ghodrati2003} presented a probabilistic seismic hazard assessment of metropolitan Tehran, the capital of Iran. The results showed that the PGA ranges from 0.27-0.46 $g$ for a return period of 475 years and from 0.33-0.55 $g$ for a return period of 950 years. 
%In this study we use background seismicity to calculate the seismic hazard.Using background seismicity, \citet{Cao1996} conducted a seismic hazard estimation study in southern California. They concluded that one could obtain similar results implementing background seismicity to those obtained by introducing zones for areas with well-recorded seismicity patterns. \citet{Lapajne1997} used spatially smoothed seismicity modeling to acquire a seismic hazard outlook in Slovenia. They defined 4 different models with different correlation distances, including a model based of the total released seismic energy. They gained an acceptable PGA in comparison with other seismic hazard approaches. \citet{Akinci2004} estimate the seismic hazard in central and northern Italy using smoothed historical seismicity. Their results showed that the smoothed seismicity approach gives reasonable regionalized results without introducing seismogenic zones. This approach has also been used in many regions with different seismicity patterns \citep{Wesson1999, Klein2001, Hamdache2008, Kalkan2009, Moschetti2014, Boyd2008}.
%
%\subsection{Analysis Method}
%\subsubsection{smoothed seismicity method}
%To calculate the seismic hazard, we use spatially smoothed seismic hazard analysis \citep{Frankel1995}. In this model, seismic events are spatially gridded to cells. We are attempting to assess the relative likelihood of moderate earthquakes ($M_w > 4.5$), which cause structural damage. According to \citet{Frankel1995}, moderate earthquakes generally occur in areas that there have been significant numbers of events of magnitude 3 and above. Therefore, these events provide a reasonable guide to where moderate earthquakes will most likely occur. Since the catalog's completeness for $M3+$ and $M4+$ is different, we use two magnitude completeness range for less than $M_ w5$ . Category 5+ and others ($M6+$ and $M7+$) assume that future $M_w 4.5$ events will occur near where they have occurred in the past. \\
%\noindent
%According to Building and Housing Research Center  \citep{BHRC2014}, $M_w5$ is considered as a threshold magnitude to structural damage. In the smoothed seismicity method, the model uses each event location as a point source, hence $M_w4.5$ could be damaging earthquake for old masonry buildings or even engineering building if it is very close to it, in this study, in order to consider the probabilistic seismic hazard of both models, we defined two models based on $M_w 5$ and $M_w 4.5$ as a threshold magnitude for seismic hazard calculation. \\
%\noindent
%\noindent
%In order to calculate the annual rate ($\lambda$) of exceeding ground motion ($u$) at a specific site, according to \citet{Frankel1995}, first we need to divide the region into cells (0.1*0.1) and count all earthquake events which are bigger than $M_{ref}$ in each cell. We use Herrmann formula \citep{Herrmann1977} to convert the cumulative (i.e. number of events bigger than $M_{ref}$) values to incremental (i.e. number of events from$M_{ref}$ to $M_{ref}+\Delta_m$) values. Using Gaussian function we smooth the number of events in each cell. The smoothed value $\tilde{n}_i$ is obtained from
%
%\begin{equation}
%\tilde{n_i}=\frac{\sum_{j} n_{j} e^{\frac{-\Delta_{ij}^{2}}{c^2}}}{\sum_{j} e^{\frac{-\Delta_{ij}^{2}}{c^2}}},
%\end{equation}
%
%\noindent
%where,  $\tilde{n}_i$  is normalized to preserve the total number of events. $\Delta{_i{_j}}$ is the distance between the $i{_t{_h}}$ and $j{_t{_h}}$ cell. The sum is calculated over cells $j$ within a distance of $3c$ of cell $i$.\\
%\noindent
%Then we discretized the magnitude and distance in uniform bins and get the total number of Normalized, smoothed events ($N_k$) inside the certain distance increment.
%Having ($N_k$) and $T$ (catalog time duration that we calculated $N_k$ within that duration), we can calculate the annual rate  $\lambda (u> u_0 )$ of exceeding ground motion according to 
%
%\begin{equation}
%\begin{split}
%\lambda(u>u_{0}) = \sum_{k}\sum_{l}10^{[log(\frac{N_{k}}{T}-b(M_l-M{_r{_e{_f}}}))]} \\
%p(u>u_0 | D_k ,M_l),
%\end{split}
%\end{equation}

%\noindent
%where $k$ and $l$ are denoted as index of distance and magnitude bin. $P(u>u_0 | D_k,M_l )$ is the probability that for an earthquake at distance $D_k$ with magnitude $M_l$, u at the site will exceed  ground motion $u_0$. This probability depends on the attenuation relation and the standard deviation (aleatory variability) of the ground motion for any specific distance and magnitude \citep{Frankel1995}

