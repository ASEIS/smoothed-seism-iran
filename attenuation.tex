
\section{Attenuation Relationship}

The last piece in our hazard analysis process is the choice of an adequate attenuation relationship, or ground motion prediction equation (GMPE). This is a highly influential factor because it controls the spatial extent to which the local occurrence of earthquakes can influence regional hazard. 

Recently, \citet{Zafarani2014} investigated the predictive capabilities of a set of nine local, regional, and next generation attenuation (NGA) GMPEs to determine their applicability for northern Iran. They evaluated GMPE predictions against data from 32 earthquakes of magnitudes $M_w$ ranging between 4.7 and 7.4. This included comparisons of PGA and response spectral accelerations (SA) computed from time-series recorded on 163 stations located at epicentral distances of up to 200 km. The evaluation was done using the likelihood (LH) and log-likelihood (LLH) methods of \citet{Scherbaum_2004_BSSA, Scherbaum_2009_BSSA}. The combined evaluation for PGA, and SA for seven periods (T) between 0.1 and 2.0 s, yielded that the best overall predictions were those of the GMPEs introduced by \citet{Ghasemi_2009_JS}, \citet{Abrahamson_2008_ES}, and \citet{Chiou2008}. For the specific case of PGA ($T=0$ s), however, the best results were those obtained with the GMPEs introduced by \citet{Kalkan2004}, \citet{Chiou2008}, and \citet{Boore2008}.

Since our hazard analysis here focuses on PGA values only, we concentrated our attention in the last set of GMPEs. In comparisons not shown here for brevity, we found that these GMPEs yielded mean predictions that were very close to each other. Furthermore, based on the LLH coefficients reported by \citet{Zafarani2014}, and using the approached of \citet{Scherbaum_2009_BSSA}, we found that a logic-tree analysis lead to weights of 0.3376, 0.3354, and 0.3270 for \citet{Kalkan2004}, \citet{Chiou2008}, and \citet{Boore2008}, respectively. These results meant that neither of them had a substantial advantage over the others. 

Based on this, we selected the GMPE proposed by \citet{Kalkan2004} for our analysis. This equation is of the form
% 
\begin{align}
	\ln \left( Y \right) =
		& \hspace{1ex} b_1 + b_2(M_w - 6) + b_3 \left( M_w - 6 \right)^{2} \nonumber \\ 
		& + b_5 \ln \left( r \right) + b_V \ln \left( V_S / V_A \right)
	\, ,
\end{align}
% 
where $Y$ is a ground motion parameter (here representing PGA), $V_A$ is a reference velocity in m/s, $V_S$ is the shear wave velocity of the site of interest. In this equation, the distance $r$ is given by
% 
\begin{equation}
	r= \sqrt{ r^2_{\mathit{cl}} + h^2 }
	\, ,
\end{equation}
% 
where $r_{\mathit{cl}}$ is the horizontal distance to the site of interest and $h$ is a reference fictitious depth, both given in km. According to \citet{Kalkan2004}, in the case of $Y$ representing PGA, the coefficients $b_1$, $b_2$, $b_3$, $b_5$ and $b_V$ are
% 
\begin{equation}
\begin{array}{lcrlcr} 
	b_1 &=&  0.393   \,,&\hspace{2em}   b_2 &=& 0.576\,,   \\
	b_3 &=& -0.107   \,,&\hspace{2em}   b_5 &=& -0.899\,,  \\
	b_V &=& -0.200   \,; 
	\nonumber
\end{array}
\end{equation}
% 
and $V_A$, $V_S$ and $h$ are
% 
\begin{equation}
\begin{array}{lcrl} 
	V_A &=& 1,112 & \mathrm{m/s}	\\
	V_S &=&   700 & \mathrm{m/s}	\\
	h   &=&  6.91 & \mathrm{km}\,,
	\nonumber
\end{array}
\end{equation}
% 
respectively. The standard deviation of the residuals $(\sigma_{\ln y})$ expressing the random variability of the ground motions is 0.612. The value of $V_S$ is assumed to represent average surface rock sites.

We should note here that even though \citet{Kalkan2004} derived the above attenuation relationship for distances $r$ up to 250 km, \citet{Zafarani2014} only used data up to 200 km in their evaluation of the various GMPEs available. Therefore, to be consistent with the latter, we set the models in the hazard analysis to compute ground motions (i.e., PGAs) at distances no greater than 200 km.

